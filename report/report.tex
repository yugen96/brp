\documentclass[a4paper]{article}

\usepackage[english]{babel}
\usepackage[utf8]{inputenc}
\usepackage{amsmath}
\usepackage{graphicx}
\usepackage{grffile}
\usepackage{caption}
\usepackage[section]{placeins}
\usepackage{flafter}
\usepackage{newunicodechar}
\usepackage{wrapfig}
\usepackage{enumitem}
\usepackage{listings}
\usepackage{color}
\usepackage{filecontents}
\usepackage{gensymb}
\usepackage{textcomp}
\usepackage[round]{natbib}
\usepackage[colorinlistoftodos]{todonotes}
\usepackage[colorlinks=true, allcolors=blue]{hyperref}

\newcommand\numberthis{\addtocounter{equation}{1}\tag{\theequation}}
\newcommand{\aap}{A\&A}
\newcommand{\nat}{Nature}
\newcommand{\araa}{Annual Review of Astronomy and Astrophysics}
\newcommand{\mnras}{Monthly Notices of the Royal Astronomical Society}
\newcommand{\pasj}{Publications of the Astronomical Society of Japan}
\newcommand{\apj}{The Astrophysical Journal}



\title{Bachelor Research Project}
\author{Bouke Jung (s1499440)}
\date{\today}


\begin{document}
\maketitle


\tableofcontents





\section{Introduction}

    \par In recent years a lot of progress has been booked in terms of understanding AGN (active galactic nuclei) feedback into the intracluster medium (ICM) and its effects on galaxy evolution and morphology. Astronomers have not only been able to distinguish different modes of energetic transfer within the feedback, but have also come up with convincing observational evidence for the presence of these modes within various galaxies and galaxy clusters \citep{Fabian2012}. \todo{more examples} Nevertheless, some problems concerning details of the models persist. For example, there remain a lot of unclarities regarding the nature of intracluster gas stabilization processes in cool-core (CC) galaxy clusters. In a paper that was recently published in the Monthly Notices of the Royal Astronomical Society\citep{Fabian2016}, A.C. Fabian et al. presented an extensive analysis of new HST data of the brightest cluster galaxy (BCG) of the Centaurus cluster, NGC4696, which suggests that there must be additional factors which play a role in the process besides the ones already known; current research points in the direction of magnetic fields, yet no direct observational evidence of such fields has ever been presented. \\ 
    
    Since it has been shown that a substantial part of observed galaxy clusters fall within the cool-core regime (ranging from 40\% to 70\% depending upon the chosen samples of observations) \citep{Hudson2010, Chen2007, Sanderson2006, Sanderson2009}, it is of great importance to acquire more insight into the details surrounding AGN-ICM interactions and cool-core galaxy clusters. Within the following few sections, I hope to shed some light on the current status of research on these topics. I'll first delve deeper into the history behind studying cool-core clusters, explaining the development of research into what used to be called the \textit{cooling flow problem}. Afterwards I'll try to summarize present-day views surrounding AGN feedback mechanisms. I'll conclude the section by highlighting the research conducted within Fabian's 2016 paper on NGC4696 and the importance of magnetic fields.



    \subsection{The cooling flow problem}
        
        When astronomers first looked at galaxy clusters in X-ray during the seventies\todo{cite Lea et al. 1973, Cowie et Binney 1977, Fabian et Nulsen 1977, Mathews et Bregman 1978}, one curious property immediately caught their eye: the \textit{intracluster medium} (ICM) at the centers of galaxy clusters was so dense, that the cooling time of the surrounding gas turned out to be lower than the Hubble parameter! This eventually lead to the conception of the so-called \textit{cooling flow} (CF) model. Essentially it proposes that the ICM at the cluster centers cools hydrostatically. The resultant cool gas would be compressed by the surrounding medium, facilitating an influx of hot gas into the cluster center in order to replace the compressed gas (hence the name cooling flow model). \\
        Although initial X-ray observations seemed to corroborate the model, subsequent observations in the optical regime revealed several problems. First of all, the observations failed to reproduce the predicted star formation rates associated with cooling flows and found different values for CO and molecular gas abundances\todo{cite McNamara et O'Connell 1989; Edge 2001}. Furthermore, researchers discovered discrepancies between the mass deposition rates determined classically by looking at mass densities and by using spectroscopy \citep{Makishima2001}. Lastly grating spectra acquired by the XMM-Newton telescope revealed that gas in central regions does not cool at the rates predicted by the cooling flow model. \\
        
        \par The collective discrepancies between observations and the cooling flow model lead astronomers to question the validity of certain assumptions that were being made traditionally. This eventually inspired a search for heating models which initially had not been allowed. Among the mechanisms that were considered are:
        
        \begin{enumerate}
            \item conduction \todo{ref Zakamska et Narayan 2003}
            \item central AGN heating due to the combined effect of cosmic ray-ICM interactions and conduction \todo{refs (see Hudson2010)}
            \item Weak shocks induced by buoyant gas bubbles \todo{refs}
            \item a combination of sound waves and conduction\todo{refs}
            \item a combination of turbulence and conduction\todo{refs}
        \end{enumerate}
        
        Another heating model which has been proposed recently on the basis of ROSAT observations\todo{refs} \citep{Dunn2006, Birzan2004}, is the release of mechanical energy by AGN driven bubbles. Researchers suggest this mode of energy transfer might be sufficient to completely quench cooling flows.\todo{refs} \\
        
        \par As the astronomical community realized the failure of the CF model, CF clusters began to be called cool-core (CC) clusters instead. Nevertheless, the distinction between CC clusters and non cool-core (NCC) clusters initially remained vague with different researchers using different definitions. For example, some authors took a central temperature drop to be definitive\todo{ref}, whereas others opted for a definition based on central cooling time (CCT) (i.e. the radiative cooling time measured within a radius of $~100$ kpc from the center of the galaxy cluster) \todo{ref} or mass deposition rates\todo{ref}. A recent study published in 2010 \citep{Hudson2010} tries to resolve this ambiguity by showing how central cooling time forms the best parameter for distinguishing CC clusters from NCC clusters. The authors propose a classification of galaxy clusters into one of the following three categories:
        
        \begin{enumerate}
            \item strong cool-core (SCC), characterized by:
                \subitem low central entropies ($\leq 30 \mathrm{h^{-\frac{1}{3}}_{71} keV cm^2}$)
                \subitem systematic central temperature drops ($\frac{T_o}{T_{vir}} \approx 0.4$)
                \subitem a brightest cluster galaxy (BCG)
            \item weak cool-core (WCC) or transitioning, characterized by:
                \subitem moderate CCT ($1 - 7.7 \mathrm{h^{-\frac{1}{2}}_{71} Gyr}$)
                \subitem elevated central entropies ($\geq 30 \mathrm{h^{-\frac{1}{3}}_{71} keV cm^2}$)
                \subitem a BCG at/near the cluster's X-ray peak ($< 50 \mathrm{h^{-1}_{71} kpc}$)
            \item non cool-core (NCC), characterized by:
                \subitem long CCT ($> 7.7 \mathrm{h^{-\frac{1}{2}}_{71} Gyr}$)
                \subitem large central entropies ($K_0 > 110 \mathrm{h^{-\frac{1}{3}}_{71} keV cm^2}$)
                \subitem flat temperature profiles or rising towards the center
        \end{enumerate}
    
        One process which likely plays a large role in determining towards which category a galaxy cluster will evolve is active galactic nuclei feedback (\textit{AGN feedback}). The next section tries to give a brief overview of the current status of research on this topic as well as on its effects on the ICM.
    
    
    
    \subsection{AGN feedback}
        Over the last couple of years astronomers have become increasingly aware of the major role played by central massive black holes and active galactic nuclei in determining the final stellar mass of a galaxy bulge. For one thing they possess the ability to accrete vast amounts of material via gravitational interaction. But conversely, they can also sweep the bulge clean of matter via the production of intense fluxes of photons and particle rays, resulting in a loss of star formation and the eventual termination of the AGN itself. The feedback process which governs the overall interaction between the active galactic nucleus and the galaxy's stellar mass in these different ways is called \textit{AGN feedback}. Recently, various studies have been conducted in order to specify the nature of AGN feedbacks\citep{Fabian2012, McNamara2012} \todo{extra references}. This has led to the identification of two main modes of energetic transfer. One mode, called the \textit{radiative mode} (also known as the quasar or wind mode), is concerned with the movement of cold gas and operates at times when the galaxy's central accreting black hole radiates according to the Eddington limit. The other mode, called the \textit{kinetic} or \textit{maintenance} mode (sometimes called \textit{radio jet} as well), involves hot gas and is typically observed in more massive galaxies containing hot halo's and powerful jets or galaxies situated at the center of galaxy clusters. \\
        This section will try to cover these different modes of energetic transfer in order to form a better understanding of the processes surrounding AGN feedback and its applicability to cool-core clusters. Before we continue however, it is important to note that, even though the central black hole is tiny compared to the overall structure of its host galaxy, its energetic influence can be remarkable. In fact, a simple consideration involving a comparison between a typical galaxy's binding energy ($E_{gal}$) and the energy released by mass accretion of its central black hole (typically $E_{BH} \approx 0.1 M_{BH} c^2$ when taking into account a radiative efficiency of 10\% for the accretion process), will reveal that $\frac{E_{BH}}{E_{gal}} > 80$. In other words: even if only a fraction of the energy released during gravitational infall reaches the interstellar gas, the AGN will have a profound effect on the evolution of its host galaxy.
        
        \subsubsection{The radiative mode}
        
            The radiative mode of AGN feedback is mainly found in young, active galaxies hosting quasars at their cores. Since dust often highly obscures the centers of these type of galaxies, finding observational evidence for the mode has been difficult. Furthermore, there remains substantial debate surrounding the mechanism which fuels it. \\
            One mechanism leans upon a consideration involving the radiation pressure of quasars emitting at the Eddington limit. According to a paper published in 1998 \citep{Silk1998}, it is possible for quasars radiating at the Eddington limit to prevent accretion into a galaxy at a maximum rate, so long as:
            
            \begin{equation}
                M_{BH} ~ \frac{f \sigma^5 \sigma_T}{4\pi G^2 m_p c}
            \end{equation}
            
            However, this formula is based only on a consideration surrounding energy which is necessary for ejecting matter and may not be sufficient for ejecting matter. Another argument based on conservation of momentum yields:
            
            \begin{equation}
                M_{BH} = \frac{f \sigma^4 \sigma_T}{\pi G^2 m_p}
            \end{equation}
            
            which is about $\frac{c}{\sigma}$ times larger. Furthermore, it also agrees better with observed black hole mass versus stellar velocity dispersion ($M_{BH} - \sigma$) relations, which might be interpreted as weak observational evidence of the presence of AGN feedback. \\
            
            \todo{finish AGN feedback}
            
            
            
            
            
            
            
            






%----------------------------------------------------------------------------------------
%	BIBLIOGRAPHY
%----------------------------------------------------------------------------------------
%to compile succesfully use:
%    pdflatex filename (with or without extensions)
%    bibtex filename (without extensions)               %compile library file
%    pdflatex filename (with or without extensions)
%    pdflatex filename (with or without extensions)
%

\clearpage
\section{References}
\renewcommand\refname{}
\bibliographystyle{plainnat}
\bibliography{brpref.bib}





\end{document}
